\section{Introduction}
\label{sec:intro}


\npar This paper will outline a \textit{Delegate Multi-Agents System (DMAS)}
solution that uses \textit{Ant Colony Optimalisation (ACO)} techniques to solve
the dynamic \textit{Pickup and Delivery Problem (PDP)}. The performance of this
solution will be compared with two other classic \textit{Multi Agent Systems
(MAS)}, \textit{ContractNet} and \textit{Gradient Field}.

\npar The considered PDP assumes that a truck can only transport one package at
the time and that new packages can randomly be added to the system. The
infrastructure will also remain the same and congestion on road segments will
not occur.

\npar Even though computing power is getting cheaper every day, not every
(cross)road can be equipped with a device, capable of communicating with other
agents. Therefore, it is assumed that all the agents are simulated in a virtual
environment (e.g. a private cloud). Truck drivers only have a GPS device at
their disposal. They receive the list of locations they have to visit from this
virtual environment.

\npar The reader is supposed to have a decent knowledge of \textit{Multi-Agent
Systems (MAS)} and the applications of \textit{Ant Colony Optimalisation (ACO)}
techniques in \textit{Delegate Multi-Agent Systems (DMAS)}.

