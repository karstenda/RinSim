\section{Evaluation}

\npar In order to evaluate the DMAS solution, it is compared to two reference
solutions: Gradient Field and Contract Net. We will describe them in a nutshell.

\npar Gradient Fields are based on electrical fields: packages and trucks have
opposite charges and will thus attract each other. However, trucks will repel
other trucks because they are able to move and have equal charges.
This will make sure that the work is divided among all trucks.

\npar In a Contract Net, package agents will broadcast the position of their
packages. (Some) Truck agents will receive the broadcasted message and repond
with an offer to pick up the package. The package agent will then evaluate all
the offers it received and inform the senders whether they won the offer or not.
When a truck receives an accepted proposal, it can still decide that there is a
better alternative. In this case, the package agent is informed of a failure.
The contract is now broken and the package will again broadcast the position of
its package.

\subsection{Experiments}

\npar As explained in \ref{sec:dmas_algorithm}, the DMAS solution will try to
plan an optimal route. As a consequence, we are particularly interested in the
performance of this system. We test this in a number of ways.

\npar First, the overall performance is measured by counting the number of
delivered packages in a fixed time window.

\npar Second, we track the time that passes between the creation of a package
and the pickup. This can serve as an indication for the responsiveness to
dynamism.

\npar Third, all packages should be delivered within a given time window.
Therefore, the lateness for every delivery is logged.

\npar Finally, to evaluate the efficiency of the chosen routes, the
total distance travelled by trucks is also measured. 

\npar All this information is gathered during a fixed-time scenario in which
a number of trucks start with an initial amount of packages. Every time a
package is delivered, a new one is added. This will prevent the trucks from
running out of packages and will not flood the system with too much packages
either.

\npar  Because of the optimal path planning, we expect to see a high thoughput
(delivered packages) and a low value for total distance travelled.

\subsection{Comparison}

%TODO

\subsection{Critical Reflection}

\npar First of all, the system manages to get the task done. A huge advantage
to this system is that it can plan ahead, whereas the Contract Net and the 
Gradient Field solution in particular do not/cannot plan ahead. Using
heuristics, the delegate MAS solution can find an optimal path throughout the
system.

\npar One of the disadvantages about this system is that it will not always find
packages at a given moment. Sometimes, it is required to navigate randomly until
packages nearbycan ``hear'' the truck agent. One way to solve this problem would
be to periodically increase the broadcast radius of truck agents (up to a
maximal range) when no packages are found in a certain time window.

\npar Another way to solve this problem could be to combine multiple solutions
into one hybrid solution. When no packages are found, a Gradient Field could be
used to spread the agents accross the physical environment. This will increase
the coverage of the map and hence the chance for packages to be heard by trucks.

\npar %TODO
